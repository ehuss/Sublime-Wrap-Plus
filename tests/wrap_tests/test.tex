Packages/LaTeX/LaTeX.sublime-syntax
===
This is a test. This is a test. This is a test. This is a test. % This is a comment
---
This is a test. This is a test. This is a test. This is a test. % This is a comment
===
This is an example of a citation starting on a new line
\cite{Newton1687}
---
This is an example of a citation starting on a new line \cite{Newton1687}
===
sample text
\textbf{sample text}
---
sample text \textbf{sample text}
===
% Must not omit this very important statement:
Very important statement
---
% Must not omit this very important statement:
Very important statement
===
We refer the reader to the proof in Lemma~\ref{lem:FLT}
% <START>Here is some text
% that I have decided
% to remove.<END>
the proof given is particularly elegant because of reasons X,Y,Z...
---
We refer the reader to the proof in Lemma~\ref{lem:FLT}
% Here is some text that I have decided to remove.
the proof given is particularly elegant because of reasons X,Y,Z...
===
<START>The quadratic form $\mN$ for the integration of the squared norm of an $m$-dimensional linear function on any simplex $\Omega$ of dimension $n$ can therefore be written as<END>
\begin{align}
\label{eq:integration_matrix}
\mN &= \Factorial{n} \, \vol{\Omega} \, \Transpose{\mP} \inp{\mH^0_n \otimes \IdentityMatrix_m} \, \mP \nonumber\\%
    &= \Factorial{n} \, \vol{\Omega} \, \mN^0~.
\end{align}
<START>The discrete $L^2$-product of piecewise-linear functions on a simplical complexes is obtained by assembling the quadratic forms $\mN$ of each simplex into an operator $\mM$ on the whole complex.<END>

Don't touch lines that start with a backslash.
---
The quadratic form $\mN$ for the integration of the squared norm of an
$m$-dimensional linear function on any simplex $\Omega$ of dimension $n$ can
therefore be written as
\begin{align}
\label{eq:integration_matrix}
\mN &= \Factorial{n} \, \vol{\Omega} \, \Transpose{\mP} \inp{\mH^0_n \otimes \IdentityMatrix_m} \, \mP \nonumber\\%
    &= \Factorial{n} \, \vol{\Omega} \, \mN^0~.
\end{align}
The discrete $L^2$-product of piecewise-linear functions on a simplical
complexes is obtained by assembling the quadratic forms $\mN$ of each simplex
into an operator $\mM$ on the whole complex.

Don't touch lines that start with a backslash.
