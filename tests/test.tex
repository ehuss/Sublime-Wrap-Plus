This is a test. This is a test. This is a test. This is a test. % This is a comment



% Must not omit this very important statement:
Very important statement


We refer the reader to the proof in Lemma~\ref{lem:FLT}
% Here is some text
% that I have decided
% to remove.
the proof given is particularly elegant because of reasons X,Y,Z...


% better for a higher number of surfaces.
The run-time of the DT method does not depend on the number of surfaces and
scales well even for a high number of surfaces.
%


\documentclass[12pt]{article}
\usepackage{amsmath}
\title{\LaTeX}
\date{}
\begin{document}
  \maketitle
  \LaTeX{} is a document preparation system for the \TeX{}
  typesetting program. It offers programmable desktop
  publishing features and extensive facilities for
  automating most aspects of typesetting and desktop
  publishing, including numbering and cross-referencing,
  tables and figures, page layout, bibliographies, and
  much more. \LaTeX{} was originally written in 1984 by
  Leslie Lamport and has become the dominant method for
  using \TeX; few people write in plain \TeX{} anymore.
  The current version is \LaTeXe.

  % This is a comment, not shown in final output.
  % The following shows typesetting power of LaTeX:
  \begin{align}
    E_0 &= mc^2                              \\
    E &= \frac{mc^2}{\sqrt{1-\frac{v^2}{c^2}}}
  \end{align}
\end{document}


\begin{itemize}
    \item  $\pi \colon Bl_J M \rightarrow M$ is the blowup along
the Jacobian ideal $J = (Jac(f), f)$ for $X$.  Note that if we
wanted to we could have instead used
$J = (Jac(f,g), f, g) \cdot (Jac(f), f)$; this would give
us a blowup that factored through both the
Nash blowup of $Z$, and the blowup that gives us Aluffi's
formula (which holds under further blowups).  $\X$, ${\cal H}$
denote the total transforms of $X$ and $H$ under the blowup $Bl_J M$,
and $E_X$ denotes the exceptional divisor over $X$.  % actually E_X = E_Y = E_Z = E? or not
jfkldskjfadlksajfklds
    \item  $i_{X \cap H}^* \pi_* = \pi_* i_{\X \cap {\cal H}}^*$,
where $i_{\X \cap {\cal H}}^*$ is the transverse intersection map
for $\X \cap {\cal H}$.
    \item  $i_{\X \cap {\cal H}}^*[\X]$ is equal to
$[\X \pitch {\cal H}]$, and this is the same class as $[\Z]$;
similarly for $i_{\X \cap {\cal H}}^*[E_X]$.
\end{itemize}


Let $X$ and $Y$ be smooth hypersurfaces in a smooth ambient variety $M$,
where $Z = X \cap Y$ is a smooth global complete intersection and
$X$ meets $Y$ transversely.  If $\nu_Z$, $\nu_X$, $\nu_Y$ denote the
normal bundles of $Z$, $X$, and $Y$ in $M$, then the transversality of
$X$ and $Y$ means that $\nu_Z = \nu_X|_Z \oplus \nu_Y|_Z$.
Moreover, the Chern class $c(\nu_Z)$ is given by $1 + Z$, where $Z$
denotes the dual of the fundamental class $[Z]$, {\it i.e.}
$Z \cap [M] = [Z]$ (and similarly for $c(\nu_X)$ and $c(\nu_Y)$).  % missing inclusions?
Using these facts we can compute the following formula for the
MacPherson-Chern class $c_*(Z)$ of $Z$ in terms of $X$ and $Y$:


% Lorem ipsum dolor sit amet, consectetur adipisicing elit, sed do eiusmod tempor incididunt ut labore et dolore magna aliqua. Ut enim ad minim veniam, quis nostrud exercitation ullamco laboris nisi ut aliquip ex ea commodo


Lorem ipsum dolor sit amet, consectetur adipisicing elit, % Comment asdf fdsa alskdjf dsfaj fkdljsafkdsj aflkdj salfk jdslkafj jd salkfjd.

Lorem ipsum dolor sit amet, consectetur adipisicing elit, % Comment asdf fdsa
                                                          % alskdjf dsfaj
                                                          % fkdljsafkdsj aflkdj
                                                          % salfk jdslkafj jd
                                                          % salkfjd.
